\documentclass{article}

\usepackage{amsmath}

\title{COMP 2804 - Assignment 1}
\author{Michael Maxwell - 101006277}
\date{February 1\textsuperscript{st}, 2017}

\addtolength{\oddsidemargin}{-1in}
\addtolength{\evensidemargin}{-1in}
\addtolength{\textwidth}{1.75in}
\addtolength{\topmargin}{-.875in}
\addtolength{\textheight}{1.75in}

\begin{document}
	\pagenumbering{gobble}
	\maketitle
	\pagenumbering{arabic}

	\begin{enumerate}
		\setcounter{enumi}{1}
		\item
			Determine the number of awesome passwords. \\
			Awesome password requirements.
			\begin{itemize}
				\item String of 10 characters
				\begin{itemize}
					\item a b c d e f g h i j k l m n o p q r s t u v w x y z
					\item 0 1 2 3 4 5 6 7 8 9
					\item ! @ \# \$ \% \& ( )
				\end{itemize}
				\item Must contain at least one digit or special character.
			\end{itemize}
			First, count the total number of passwords. The character set we're working with is of length 44. \\
			Then we will subtract all the password without any digits or special characters by the Complement Rule. \\
			\\
			$U = 44^{10}$ ``all 10 character passwords'' - by the Product Rule. \\
			$A = 26^{10}$ ``all non-awesome passwords'' - by the Product Rule.
			\begin{align*}
				P &= U - A \\
				P &= 44^{10} - 26^{10}
			\end{align*}
		\item
			Determine the number of functions. \\
			$f: \{1, 2, 3, 4\} \rightarrow \{a, b, c, \ldots, z\}$ such that $A: f(1) = f(2)$, or $B: f(3) = f(4)$, or $C: f(1) \neq f(3)$
			\begin{align*}
				\left|A\right| &= 17576&\text{ every function has 26 choices except }&f(2)\text{ has 1.} \\
				\left|B\right| &= 17576&&f(4)\text{ has 1.} \\
				\left|C\right| &= 439400&&f(3)\text{ has 25.} \\
				\left|A \cap B\right| &= 676&&f(2), f(4)\text{ have 1.} \\
				\left|B \cap C\right| &= 16900&&f(3)\text{ has 25, }f(4)\text{ has 1.} \\
				\left|C \cap A\right| &= 16900&&f(3)\text{ has 25, }f(2)\text{ has 1.} \\
				\left|A \cap B \cap C\right| &= 650&&f(3)\text{ has 25, }f(2), f(4)\text{ have 1.}
			\end{align*}
			\begin{align*}
				\left| A \cup B \cup C\right| &= \left|A\right| + \left|B\right| + \left|C\right| - \left|A \cap B\right| - \left|B \cap C\right| - \left|C \cap A\right| + \left|A \cap B \cap C\right| \text{by Inclusion-Exclusion} \\
				\left|A \cup B \cup C\right| &= 17576 + 17576 + 439400 - 676 - 16900 - 16900 + 650 \\
				&= 440726
			\end{align*}
		\item
			Let $n \geq 1$ be an integer. \\
			\begin{itemize}

			\item
				Assume that $n$ is odd. \\
				Determine the number of bit strings of length $n$ that contain more 0's than 1's. \\
				There's no chance that the number of 1's = number of 0's since $n$ is odd. \\
				The total number of bit strings of length $n$ that can be created is $2^n$ by the Product Rule. \\
				The number of bit strings that have more 0’s than 1’s and vice versa will be equal. \\
				Therefore, the number of bit strings of length $n$ that contain more 0's than 1's equals\\
				\begin{equation*}
					2^n \div 2 = 2^{n - 1}
				\end{equation*}
			\item
				Assume that $n$ is even. \\
				So there must be $n \div 2$ many 0's and 1's in the bit string. Meaning it can be solved with a binomial coefficient. Choose $n \div 2$ places in a bit string of length $n$ to place 0's. The other spots will be filled with 1's. \\
				\begin{equation*}
					{n \choose n \div 2}
				\end{equation*}
				This is a binomial coefficient summation. Since binomial coefficients are symmetric the sum can be taken from $[1, n \div 2 - 1]$.
				Choose $k$ positions to put 1's and put 0's in the other remaining positions.
				The number of bit strings of length $n$ that contain more 0's than 1's equals
				\begin{equation*}
					\sum_{k = 1}^{n \div 2 - 1} {n \choose k}
				\end{equation*}
				Argue that the following binomial coefficient is an even integer.
				\begin{equation*}
					{n \choose n \div 2}
				\end{equation*}
				$n \geq 2$ now since $n$ is an even integer $\geq 1$. \\
				Breaking down the binomial coefficient to it's expression in terms of factorials will help prove ${n \choose n \div 2}$ is an even integer.
				\begin{equation*}
					{n \choose n \div 2} = \frac{n!} {\frac{n}{2}! \times (n - \frac{n}{2})!} = \frac{n!}{(\frac{n}{2}!)^2}
				\end{equation*}
				$n$ is even because $n \geq 2$, therefore it must be a multiple of 2. \\
				$n \div 2$ is even because an even number divided by an equal or smaller even number returns an even number. \\
				Proceeding to square and divide only even numbers will remain even.
			\end{itemize}
		\item
			Let m, n, k, l be integers such that $m \geq 1$, $n \geq 1$, and $1 \leq l \leq k \leq n$. \\
			\begin{itemize}
				\item $m$ types of beer
				\item $n$ types of cider
				\item $k$ pints total
				\item $l$ pints of cider
			\end{itemize}
			Elisa has to choose $k - l$ beers from the $m$ types on tap and choose $l$ ciders from the $n$ types available. A binomial coefficient can be used to calculate those, then because the order matters it will be multiplied by $k!$. 
			\begin{equation*}
				{m \choose k - l} \times {n \choose l} \times k!
			\end{equation*}
		\item
			A store sells $n$ different types of IPA and $n$ different types of wheat beer. \\
			Where $n \geq 2$ is an integer. \\
			Prove that
			\begin{equation*}
				{2n \choose 2} = 2{n \choose 2} + n^2
			\end{equation*}
			Starting with the LHS, there is 3 different ways to choose 2 objects from 2 set of n things.
			\begin{enumerate}
				\item Buy 2 IPA's
				${n \choose 2}$
				\item Buy 2 wheat beers
				${n \choose 2}$
				\item Buy 1 IPA and 1 wheat beer
				${n \choose 1} \times {n \choose 1} = n^2$
			\end{enumerate}
			The left side is equal to $2 \times {n \choose 2} + n^2$ \\
			Therefore the equation is true, since the LHS = RHS. \\
		\item
			A string consisting of characters is called \emph{cool} if exactly one character in the string is equal to $x$ and each other character is a digit. \\
			Let $n \geq 1$ be an integer. \\
			\begin{itemize}
				\item Determine the number of cool strings of length $n$ \\
				\begin{equation*}
					n \times 10^{n - 1}\text{ : by choosing a location for $x$ then the remaining n - 1 positions have 10 choices each.}
				\end{equation*}
				\item Let $k$ be an integer with $1 \leq k \leq n$. Determine the number of cool strings of length $n$ that contain exactly $n - k$ many 0's. \\
				\begin{equation*}
					{n - 1 \choose n - k} \times 9^{k - 1}
				\end{equation*}
				\item Use the above two results to prove that
				\begin{equation*}
					\sum_{k = 1}^{n} k {n \choose k} \times 9^{k - 1} = n \times 10^{n-1}
				\end{equation*}
				The number of cool strings of length $n$ that contain exactly $n - k$ many 0's can be used to calculate all cool strings. Since the sum of the cool strings with $[0, n - 1]$ ($n - 1$ because one character still needs to be $x$) zeros is equal to the total number of cool strings.
			\end{itemize}
		\item
			Determine the number of elements $x$ in the set $\{1, 2, 3, \ldots, 99999\}$ for which the sum of the digits in the decimal representation of $x$ is equal to 8. \\
			The numbers in the set will be encoded into bit strings of length $n + k - 1 = 12$. \\
			Where $n = 8$ corresponding to the sum of the digits required, $k = 5$ corresponding to how many digits. \\
			\begin{equation*}
				{12 \choose 4} = 495
			\end{equation*}
		\item
			Let $n \geq 2$ be an integer. \\
			Let $G = (V, E)$ be a graph whose vertex set $V$ has size $n$ and whose edge set $E$ is non-empty. \\
			The degree of any vertex $u$ is defined to be the number of edges in $E$ that contain $u$ as a vertex. \\
			Prove that there exists at least two vertices in $G$ that have the same degree. \\
			\\
			When $G$ is a connected graph, each vertex will have a degree between $[1, n - 1]$ because a connected graph's vertices need to have degree at least 1 and the most vertices it connect to is every vertex but itself, so $n - 1$. \\
			Where $G$ is a disconnected graph, each vertex will have a degree between $[0, n - 2]$ because for it is okay for a vertex to have degree 0 in disconnected graphs and the most edges it can have is $n - 2$ because if it were $n - 1$ then it would be connected to every other vertex, making it connected. \\
			Now, we have $n$ vertices in the vertex set $V$ and $n - 1$ options for vertex $u$'s degree. Therefore there must be 2 vertices with the same degree by the Pigeonhole principle.
	\end{enumerate}	
\end{document}